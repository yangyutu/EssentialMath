
\begin{refsection}
\startcontents[chapters]	
\chapter{Differential Geometry \& Riemannian Manifolds}\label{ch:differential-geometry--riemannian-manifolds}
%\minitoc

\printcontents[chapters]{}{1}{}

\section{Basic theory of curves}
\begin{definition}[curve]\index{curve}
A parameterized smooth curve in $\R^n$ is a differentiable map $r: I\subset \R \to \R^n$ from an open interval of the real line to $\R^n$, such that
$$r(t) = (x_1(t),x_2(t),...,x_n(t))$$
where $x_1(t),x_2(t),...,x_n(t) \in C^{\infty}$ are the differentiable components or coordinate functions of $r(t)$.
\end{definition}


\begin{example}\hfill
\begin{itemize}
    \item $n=2:$ the curve $r(t)=(x(t),y(t))\in \R^2$ is called a plane curve.
    \item $n=3:$ the curve $r(t)=(x(t),y(t),z(t))\in \R^2$ is called a space curve.
\end{itemize}
\end{example}


\begin{definition}[regularity]
A parametrized smooth curve $r(t): I\to \R^n$ is called regular if $\dot{r}(t) \neq 0,\forall t\in I$.
\end{definition}


\begin{definition}[unit tangent vector, normal]
	Let $x(t)$ be a smooth curve, then its unit tangent vector at $x(t_0)$is defined as
	$$T(t_0) \triangleq \frac{\dot{x}(t_0)}{\norm{\dot{x}(t_0)}};$$
	it normal at $x(t_0)$is defined as
	$$N(t_0) \triangleq \frac{\dot{T}(t_0)}{\norm{\dot{T}(t_0)}}.$$
\end{definition}

\begin{lemma}[normal is perpendicular to tangent]
	$$N(t)\perp T(t)$$
or equivalently,
	$$N(t)\cdot T(t) = 0.$$
\end{lemma}
\begin{proof}
Note that
$$N(t)\cdot T(t) = \frac{dT}{dt}\cdot T/\norm{N} = \frac{1}{2\norm{N}}\frac{d\norm{T}^2}{dt} = 0.$$
where we use the fact that $\norm{T}^2 = 1$.
\end{proof}

\subsection{Length of a curve}



\begin{theorem}[length of a curve]\index{curve length}\cite[26]{kreyszig2013differential}\label{ch:differential-geometry--riemannian-manifolds:th:curvelength}
Let $c:[a,b]\to \R^3$ be a smooth, regular parametrized curve. Then length of $c$ is defined to be
$$l = \int_{a}^b \sqrt{(\frac{dc}{dt})^T\frac{dc}{dt}}dt,$$
where
$dc/dt \in \R^3.$
\end{theorem}
\begin{proof}
Can be proven by starting from the definition that the length of a curve is the sum of the tiny segments. 
\end{proof}

\begin{lemma}[curve length is independent of parameterization]\cite[27]{kreyszig2013differential}
Let $c:[a,b]\to \R^3,t\in [a,b]$ be a smooth, regular parametrized curve. Let $t':[a,b]\to [a',b']$, such that $c(t'):[a',b']\to \R^3$ is another smooth and regular parametrization of the same curve.  Then length of $c$ is defined to be
$$l = \int_{a}^b \sqrt{(\frac{dc}{dt})^T\frac{dc}{dt}}dt = \int_{a'}^{b'}\sqrt{(\frac{dc}{dt'})^T\frac{dc}{dt'}}dt'.$$
\end{lemma}
\begin{proof}
	$$l = \int_{a}^b \sqrt{(\frac{dc}{dt})^T\frac{dc}{dt}}dt = \int_{a'}^{b'}\sqrt{(\frac{dc}{dt'}\frac{dt'}{dt})^T\frac{dc}{dt'}\frac{dt'}{dt}}\abs{\frac{dt}{dt'}}dt' = \int_{a'}^{b'}\sqrt{(\frac{dc}{dt'})^T\frac{dc}{dt'}}dt'.$$
\end{proof}

\begin{definition}[arc length and its parameterization]\index{arc length}\cite[28]{kreyszig2013differential}
Let $x(t) \in \R^3, t\in [t_0,b]$ be a curve, then the function
$$s(t) = \int_{t_0}^t \norm{\frac{dx}{dt}} dt $$ 
is called the \textbf{ arc length of the curve}.

The parameterization of a curve by its arc length is known as \textbf{natural parameterization}. And we have
$$ds = \norm{\frac{dx}{dt}} dt$$
and 
$$\frac{dx(s)}{ds} = \frac{dx(s)}{dt}\frac{dt}{ds} = \frac{\frac{dx}{dt}}{\norm{\frac{dx}{dt}}}.$$
\end{definition}

\begin{definition}[unit tangent vector, alternative]\cite[29]{kreyszig2013differential}
	Let $x(s)$ be a curve with natural parameterization, then its unit tangent vector is defined as
	$$\frac{dx}{ds},$$
	which is of unit length via 
	$$\frac{dx(s)}{ds} = \frac{dx(s)}{dt}\frac{dt}{ds} = \frac{\frac{dx}{dt}}{\norm{\frac{dx}{dt}}}.$$
\end{definition}

\subsection{Curvature}
\begin{definition}[curvature of a curve]\index{curvature of a curve}\index{curvature}
The curvature of a curve $t(s)$ with natural parameterization is defined as
$$\kappa(s) = \norm{\frac{dt(s)}{ds}}_2 = \norm{\frac{d^2x}{ds^2}}_2$$
The radius of curvature is defined as
$$\rho(s) = \frac{1}{\kappa(s)}.$$
\end{definition}



\section{Basic theory of surfaces}
\begin{definition}[patch]\cite[134]{krim2015geometric}
A patch is smooth map $r:U\subset \R^2 \to \R^3$ from an open subset $U$ of real plane to $\R^3$, such that 
$$r(u,v) = (x(u,v),y(u,v),z(u,v))$$
where the differentiable coordinate functions $x(u,v),y(u,v),z(u,v)$, which are functions from $U\to \R$ and in $C^{\infty}$. 
\end{definition}

\begin{remark}
A patch is essentially a smooth map.
\end{remark}




\begin{lemma}[linear independence]
The vector $r_u$ and $r_v$ are linearly independent at a point $(u,v)$ if any one of the following holds:
\begin{itemize}
    \item 
    $$ 
    det\begin{pmatrix}
        r_u\cdot r_u ~ r_u\cdot r_v \\
        r_u \cdot r_v ~ r_v \cdot r_v
        \end{pmatrix}
    \neq 0 $$
    \item $r_u\times r_v \neq 0$
\end{itemize}
\end{lemma}
\begin{proof}
(1) Note that $det(*) = (r_u\cdot r_u)(r_v\cdot r_v) - (r_u\cdot r_v)^2 \geq 0$ is the Cauchy inequality and the equality only holds when $r_u=\lambda r_v$( which is linear independence); (2) note that $\norm{r_u\times r_v}^2 = det(*)$ 
\end{proof}





\begin{definition}[regular patch]\cite[135]{krim2015geometric}
A patch $r:U\subset \R^2 \to \R^3$(where $U$ is an open set) is called a \textbf{regular patch} if $r_u\times r_v \neq 0$
\end{definition}






\begin{definition}[regular surface]\cite[135]{krim2015geometric}
A subset $M$ of $\R^3$ is called a regular surface if, for \textbf{any point} $p$ in $M$ there is an open neighborhood set $V$ of $p$ in $\R^3$ and a map $r:U\to V\cap M$, where $U$ is an open set in $\R^2$, and $r(u,v)=(x(u,v),y(u,v),z(u,v))$, such that 
\begin{itemize}
    \item $r$ is smooth
    \item $r$ is a homeomorphism
    \item $r$ is regular
\end{itemize}
\end{definition}


\subsection{Tangent space}
\begin{definition}[tangent plane]
The tangent plane $T_pM$ at a point $p$ of a regular surface $M$ is the set of tangent vectors at $p$ of all curves in $M$ passing through $p$.
\end{definition}

\begin{remark}[tangent plane form a subspace]
The tangent plane $T_pM$ is a vector subspace and it can be spanned by $\{r_u,r_v\}$
\end{remark}




\subsection{Vector fields on surface}


\begin{definition}[vector fields and tangent vector fields on surfaces]
A vector field $F$ on a surface $M$ is a function that assigns to each point $p$ of $M$ a vector $F(p)\in \R^3$. 

The vector field $F$ is called a \textbf{tangent vector field} $F(p)\in T_pM$
\end{definition}





\begin{definition}[geodesic curve]\cite[147]{krim2015geometric}
A unit-speed curve $r(t)$ on a surface $M$ is called a geodesic if it has zero geodesic curvature at every point.
\end{definition}

\begin{definition}[first fundamental form]\index{first fundamental form}
Let $g_{11} = E, g_{12} = g_{21} = F, g_{22} = G$, then
$$ds^2 = E(du)^2 + 2Fdudv + G(dv)^2$$
\end{definition}

\begin{remark}[implications]\hfill
\begin{itemize}
    \item The first fundamental form enables us to measure arc length, angles and areas on a surface.
    \item We can view a surface as a two dimensional Riemannian manifold, with the metric $g$ being the Riemannian metric.
\end{itemize}
\end{remark}


\begin{definition}[second fundamental form]\index{second fundamental form}

\end{definition}


\section{Manifolds}
\subsection{Topological manifold}
 \begin{definition}[homemorphism]\index{homemorphism}
	Suppose $f: X\rightarrow Y$ is a bijective function between topological spaces $X,Y$. If both $f,f^{-1}$ are continuous, then $f$ is called a \emph{homeomorphism}. Two topological spaces $X,Y$ are said to be \emph{homeomorphic} if there exist a homeomorphism between them.
\end{definition}

\begin{definition}[topological manifold]
A topological space $M \subseteq \mathbb{R}^m$ is a manifold if for every $x \in M$, an open set $O \subset M$ exists such that:
\begin{enumerate}
	\item $x \in O$
	\item $O ~\text{is homeomorphic to}~ \mathbb{R}^n$
	\item $n$ is fixed for all $x \in M$
\end{enumerate}
The fixed $n$ is the dimension of the manifold.	
\end{definition}


\begin{remark}
A manifold is a topological space that locally resembles Euclidean space near each point.
\end{remark}


\subsection{Charts, Atlases \& smooth manifold}
\subsubsection{Charts}
\begin{definition}[chart]
A \emph{chart} on a manifold $M$ is a subset $U\subseteq M$ together with a bijective map $\phi:U\rightarrow \phi(U)\subseteq \mathbb{R}^n$. We usually denote $\phi(m)$ by $(x^1,x^2,...,x^n)$ and call $x^i$ the coordinates of the point $m \in U$. 	
\end{definition}

\begin{definition}[transition maps, overlap maps, compatibility]
Given two charts $(U_1,\phi_1),(U_2,\phi_2)$, then we get \emph{overlap} or \emph{transition} maps
$$\phi_2 \circ \phi_1^{-1}:\phi_1(U_1\cap U_2) \rightarrow \phi_2(U_1\cap U_2)$$ and 
$$\phi_1 \circ \phi_2^{-1}:\phi_2(U_1\cap U_2) \rightarrow \phi_1(U_1\cap U_2).$$	
Two charts $(U_1,\phi_1),(U_2,\phi_2)$ are called compatible if the overlap maps are smooth.
\end{definition}

\begin{remark}
(The inverse mapping is understood as: $\phi_1(U_1 \cap U_2) \subseteq \mathbb{R}^n$, then $\phi_1^{-1}$ will map the open set $\phi_1(U_1 \cap U_2)$ to $\phi_1(U_1 \cap U_2)$, then $\phi_2$ will act on this set. )(A smooth function is a function that has derivatives of all orders everywhere in its domain.) 	
\end{remark}


\subsubsection{Atlases}
\begin{definition}[Alases]
An Atlases for a manifold $M$ is a collection $\mathcal{A}=\{U,\phi \}$ of charts with the properties that $M = \cup_{a\in \mathcal{A}} U_a$ and that, whenever $U_a \cap U_b \neq \emptyset$, $\phi_a,\phi_b$ are compatible maps.	
\end{definition}


\begin{definition}[smooth atlas]
A \emph{smooth atlas} on topological manifold $M$ is a collection of continuous coordinate system on $M$ such that the following two conditions hold:
\begin{enumerate}
    \item every point of $M$ belongs to the coordinate patch of at least one of these coordinate systems
    \item the coordinate systems in the atlas are smoothly compatible with each other.
\end{enumerate}
\end{definition}

\subsubsection{smooth manifold}
\begin{definition}[smooth manifold]
A \emph{smooth manifold} $(M,\mathcal{A})$ consists of a topological manifold $M$ together with a maximal smooth atlas $\mathcal{A}$ of coordinate systems of $M$. A \emph{smooth} coordinate system $(x^1,x^2,...,x^n)$ on $M$ is a coordinate system belonging to the maximal smooth atlas $\mathcal{A}$. 
\begin{mdframed}
\textbf{Manifold with boundary is not a manifold}
A manifold with boundary is not a manifold any more, since subsets contains points on the boundary are not open. 
\end{mdframed}
\end{definition}


\section{sub-manifolds}
\begin{definition}
Let $M$ be a subset of a $k-$dimensional smooth manifold $N$. We say that $M$ is a smooth embedded sub-manifold of $N$ of dimension $n$ if, given any point $m$ of $M$, there exists a smooth coordinate system $(u^1,u^2,...,u^k)$ defined over some open set $U$ in $N$, where $m \in U$, with the property that $$M \cap U = \{p\in U: u^i(p)=0, i=n+1,n+2,...,k\} (k > n).$$
\end{definition}

\begin{remark}
A sub-manifold of a manifold is a manifold itself.	
\end{remark}


\subsection{ Sub-manifold of Euclidean space}

\begin{definition}\cite{walschap2015multivariable}
A subset $M \subset \mathbb{R}^{n+k}$ is said to be an $n-$dimensional submanifold of $\mathbb{R}^{n+k}$ if each $p\in M$ admits an open neighborhood $U \in \mathbb{R}^{n+k}$ and there exists a one-to-one differential map $h: V\rightarrow \mathbb{R}^{n+k}, V\subset \mathbb{R}^n $ defined on some open set $V$, such that
\begin{enumerate}
    \item $h$ has maximal rank(=n) everywhere;
    \item $h(V) = U\cap M$ and
    \item $h^{-1}: U\cap M \rightarrow V$ is continuous.
\end{enumerate}
A subset $M$ for which the first two conditions hold but not the third is called an \emph{immersed submanifold}. Note that any open set $U$ in $\mathbb{R}^n$ is a trivial example of sub-manifold of $\mathbb{R}^n$ with dimension $n$.
\end{definition}

\subsection{Smooth mapping between smooth manifold}
\begin{definition}
Let $M_1$ and $M_2$ be smooth manifolds of dimension $n_1$ and $n_2$. A continuous map $f:M_1\to M_2$ is said to be smooth if, for any coordinate charts $(U,\phi)$ on $M_1$ and $(V,\psi)$ on $M_2$, the composition function $\psi\circ f \circ \phi^{-1}: \phi(U)\to \psi(V)$ is a smooth map. 
\end{definition}


\begin{remark}[functions defined on a manifold]
Let $M$ be a smooth n-manifold. A function $f:M\to \R$ at a point $p\in M$ can be evaluated using local coordinate system, i.e. a coordinate chart $(U,\phi)$ as the composition $F_U=f\circ \phi^{-1}: \phi(U)\to \R,\phi(U)\subset \R^n$.\\
Note that a function $f$ in nature express the associative relationship between two sets, and usually does not have a parameteric form if $M$ is not parameterized. 	
\end{remark}
\begin{mdframed}
\end{mdframed}

\begin{example}
Let $M$ be a regular curve described by the mapping $r: I \to \R^3$. A function defined on $f:M\to \R$ evaluated at $p\in M$ can be evaluated at local coordinate as $f(r^{-1}(p)):I\to \R$.
\end{example}


\section{Tangent space on Manifold}
\subsection{Tangent space of $\mathbb{R}^n$}
\begin{definition}[tangent space of $\R$]
The tangent space of $\mathbb{R}^n$ at $p\in \mathbb{R}^n$ is the set
$$\{\mathbb{R}_p^n\}=\{p\}\times \mathbb{R}^n=\{(p,u)|u\in \mathbb{R}^n\}.$$	

An element of $\{\mathbb{R}_p^n\}$ is called a tangent vector at $p$. $\mathbb{R}_p^n$ is a vector space with the operations
$$(p,u)+(p,v)=(p,u+v),a(p,u)=(p,au),a\in \mathbb{R},u,v\in \mathbb{R}^n.$$
\end{definition}

\begin{definition}[isomorphism between tangent spaces]
There is a canonical isomorphism $\mathcal{I}_p:\mathcal{R}^n \rightarrow \mathcal{R}_p^n$ that maps $u$ to $(p,u),u\in \mathbb{R}^n$. For any $p,q\in \mathbb{R}^n$, the map
$$\mathcal{I}_q \circ \mathcal{I}_p^{-1}: \mathbb{R}_p^n \rightarrow \mathbb{R}_q^n$$
$$(p,u) \rightarrow (q,u)$$
is an isomorphism between tangent space $\mathbb{R}_p^n $ and $\mathbb{R}_q^n$, called \emph{parallel translation}.	
\end{definition}

\begin{definition}[vector field on $\mathbb{R}^n$]
	A vector field on an open set $U \subset \mathbb{R}^n$ is a map $X$ that assigns to each $p \in U$ a tangent vector $X(p) \in \mathbb{R}_p^n$ at $p$.
\end{definition}

\subsection{Tangent space on a submanifold $M$ in $\R^n$}
\begin{definition}[tangent space on submanifolds]\cite{walschap2015multivariable}
	Let $M$ be an $n-$dimensional submanifold of Euclidean space $\mathbb{R}^{n+k}$, the tangent space $M_p$ of $M$ is the collection of velocity vector $\dot{c}(0)$ of all curves $c: I \rightarrow M$ defined on some open interval $I$ containing 0 such that $p=c(0)$. 
\end{definition}

\begin{remark}
	Note that $M$ is a subset of  $\mathbb{R}^{n+k}$, but it is not a open subset. The tangent space $M_p$ is a subset and subspace of $\mathbb{R}_p^{n+k}$. The tangent space has dimension of $n$. Note that if we treat $\mathbb{R}^n$ as a manifold, and then for every $p$, the tangent space based on the above definition is $\mathbb{R}_p^{n}$, which is equilvalent to the defintion of the tangent space on $\mathbb{R}^n$.	
\end{remark}
\subsection{Derivatives in terms of tangent space}
If $f: U \rightarrow \mathbb{R}^m$ is a differential map on an open set $U \in \mathbb{R}^n$, the derivative of $f$ at $p \in U$ is the linear transformation
$$f_{*p}: \mathbb{R}_p^n \rightarrow \mathbb{R}_{f(p)}^m$$
$$(p,u) \rightarrow (f(p),Df(p)u)$$
where $Df(p)$ is the Jacobian matrix at $p\in U$, and the only difference between $f_*$ and $Df$ is that the former includes the base point of the vector, the latter does not.



\subsection{Integral curve of a vector field on $\mathbb{R}^n$}
An integral curve of a vector field $X$ is a curve $c$ satisfies $\dot{c}=X\circ c$, where $X$ is interpreted as an operator.



\subsection{Tangent vector on manifold}
A tangent vector $X_m$ at the point $m \in M$ can be regarded as an operator, associating a real number $X_m[f]$ to any smooth real-valued function $f: M\rightarrow \mathbb{R}$ defined around $m$. The quantity $X_m[f]$ is referred to as the directional derivative of the function along vector $X_m$. If $f$ and $g$ are smooth real-valued functions defined around $m$ and if $f=g$ on some open set $V$ containing the point $m$ then $X_m[f] = X_m[g]$.\\
All the tangent vectors at the point $m$ form a linear vector space, referred as tangent space to $M$ at the point $m$. Let $(x^1,x^2,...,x^n)$ be a smooth coordinate system around the point $m$ chosen such that $x^i(m) = 0, \forall i$. Let $X_m$ be a tangent vector at the point $m$. Then $$X_m = a^1 \frac{\partial}{\partial x^1}|_m + a^2 \frac{\partial}{\partial x^2}|_m + ... + a^n \frac{\partial}{\partial x^n}|_m,$$ where $a^i = X_m[x^i]$(note here that $x^i$ is a function that maps a point in $m$ to the $i$th coordinate in $\mathbb{R}^n$, and $\frac{\partial x^i}{\partial x^j} = \delta_{i,j}$).

\subsection{Interpretation of tangent vector in tangent space of $T_p(\mathbb{R}^n)$}
\begin{enumerate}
    \item a tangent vector $X_p$ can be interpreter as an operator acting on functions defined at $p$. The basis for the tangent vector is $\{\frac{\partial}{\partial x^1},\frac{\partial}{\partial x^2},... \}$
    \item The dual basis for the cotangent space $T_p^*{\mathbb{R}^n}$ is 1-forms $\{dx^1,dx^2....\}$, and relates to basis as $$dx^i(\frac{\partial}{\partial x^j}) = \delta_{ij}$$
    \item The differential of a function $f$ defined on $\mathbb{R}^n$ is given as in 1-form: 
    $$df = \sum \frac{\partial f}{\partial x^i} dx^i$$
    \item Consider a submanifold $M$ in $\mathbb{R}^m$ with the coordinate chart $\phi:M \rightarrow \mathbb{R}^n$, where $\phi(p) = (x^1,x^2,...,x^n), p\in M$. Then the basis for tangent space at $p$ can be express as $\{\frac{\partial \phi^{-1}}{\partial x^1},\frac{\partial \phi^{-1}}{\partial x^2},... \}$ (\textbf{recalled the tangent plane in 2d surface patch}), where each column is vector of $m$ dimension. For a function defined on $M$, more formaly we should write as $f\circ \phi^{-1}(x_1,x_2,...) = \tilde{f}(x_1,x_2,...)$ when we use local coordinate system. When we take a directional derivative at $p$ with direction $v= X^i \frac{\partial \phi^{-1}}{\partial x^1} \in T_pM$, we can write
    $$\frac{d (f\circ \phi^{-1}(\phi(p) + t v))}{dt} = v^i\frac{\partial f}{dx^i}$$
    
\end{enumerate}
\subsection{partial derivative respect to local coordinate system}
Let $m$ be a point of the coordinate patch $U$. Given any smooth real-valued function $f$ defined around $m$, we denote by $\partial f / \partial x^i$ the $i$th partial derivative of the function with respect to the coordinate system $(x^1,x^2,...,x^n)$, defined by 
$$\frac{\partial f}{\partial x^i}|_m = \frac{\partial (f\circ \phi^{-1})}{\partial t^i}|_{(t^1,...t^n) = \phi(m)},$$
where $(t^1,t^2,...,t^n)$ is the standard Cartesian coordinate system on $\mathbb{R}^n$, $\phi$ is a bijective mapping from $U$ to $\mathbb{R}^n$.


\section{Topic: variable transformation for function defined on manifolds}
Given a function defined on manifold $X$, how will the function and its derivatives' parameterizations change if we change the parameterization of the manifold $X$?
For example, consider $\mathbb{R}^2$, which can be parameterized by $(x,y)$ or $(r,\theta)$. The relationships between $(x,y)$ and $(r,\theta)$ can be thought as 1-1 differential maps. A infinitestimal tangent vector in $(x,y)$ space can be mapped to a infinitestimal tangent vector in $(r,\theta)$ space using Jacobian matrix, for example\cite{chirikjian2011stochastic}, $$(dx,dy)^T = \frac{(x,y)}{(r,\theta)}^T (dr,d\theta)^T$$ For a function defined on $(x,y)$, its derivative is obtained as $$(f_x,f_y)^T = \frac{(r,\theta)}{(x,y)}^T (f_r',f_{\theta}')^T$$ 



\section{Co-variant derivative }\cite{Shifrin2015}
Give a vector field $X$ and $V \in T_pM$, we define the covariant derivative
$$\nabla_V X = (D_V X)^{\parallel} = \text{the projection of} D_V X \text{onto} T_pM$$
$$\nabla_V X = D_V X - (D_V X \cdot n)n$$
The symbol $D_V X$ is defined as: if $V \in T_pM$, we choose a curve $\gamma$ with $\gamma(0) = p$ and $\gamma'(0) = V$, and we set $D_VX = (X\circ \gamma)'(0)$, where $(X\circ \gamma)(t)$ is the vector field along the curve $\gamma$.\\
Given a curve $\gamma \in M$, we say the vector field $X$ is covariant constant or parallel along $\gamma$ if $\nabla_{\gamma'(t)}X = 0$ for all $t$.
We say a curve $\gamma \in M$ is a geodesic if its tangent vector is parallel along the curve, $\nabla_{\gamma'}\gamma = 0$ 



\section{Basic Riemannian geometry}

\begin{definition}[Riemannian metric]\index{Riemannian metric}\cite[196]{mcinerney2013first}
Let $U\subset \R^n$ be a domain. A Riemannian metric on $U$ is a smooth rank 2 tensor field $g$ satisfying two properties:
\begin{itemize}
    \item $g$ is symmetric, and for all $p\in U$, and all tangent vectors $X_p,Y_p \in T_pU$, $$g_p(X_p,Y_p) = g_p(Y_p,X_p)$$
    \item $g$ is positive definite: for all $p\in U$, and all tangent vectors $X_p \in T_p U$, 
    $$g_p(X_p,X_p) \geq 0$$
    with $g_p(X_p,X_p) = 0$ if and only if $X_p = \bm{0}_p$(the zero vector at $T_pU$) 
\end{itemize}
In other words, $g$ is a family(parametrized by $p\in U$) of inner product function defined as $T_pU \times T_pU \to \R$.
\end{definition}

\begin{remark}[representing the metric by a matrix]
It is usually convenient to express the metric tensor $g$ by \textbf{a symmetric and positive semi-definite matrix} $G$ such that $$g(X,Y) = X^TGY$$
Note that $M$ is a function of $p\in U$.
\end{remark}


\begin{definition}[Riemannian space]\index{Riemannian space}
A Riemannian space is a domain $U\subset \R^n$ equipped with a Riemannian metric $g$, and is denoted by $(U,g)$.
\end{definition}

\begin{definition}[length of a curve in Riemannian space]\index{curve length}
Let $(U,g)$ be a Riemannian space and let $c:[a,b]\to U$ be a smooth, regular parametrized curve. The length of $c$ is defined to be 
$$l = \int_a^b [g_{c(t)}(\dot{c(t)},\dot{c(t)})] dt $$
\end{definition}


\begin{remark}
The definition of a curve in the Riemannian space is a straight forward generalization of the concept of curves in Euclidean space $\R^3$.\\
(\autoref{ch:differential-geometry--riemannian-manifolds:th:curvelength}):Let $c:[a,b]\to \R^3$ be a smooth, regular parametrized curve. Then length of $c$ is defined to be
$$l = \int_{a}^b \sqrt{(\frac{dc}{dt})^T\frac{dc}{dt}}dt,$$
where
$dc/dt \in \R^3.$
\end{remark}



\subsection{Pseudo-Riemannian manifold}
A pseudo-Riemannian manifold, also called semi-Riemannian manifold is a generalization of a Riemannian manifold in which the metric tensor is non-degenerate, smooth, symmetric. The positive symmetry of the metric tensor is not required.  

\subsection{Exponential map}\index{Exponential map}
Let $M$ be a differential manifold and $p$ a point of $M$. An affine connection on $M$ allows one to define the notion of a deodesic through point $p$. Let $v \in T_pM$ be a tangent vector to the manifold at $p$. Then there is a unique geodesic $\gamma_v$ satisfying $\gamma_v(0) = p$ with initial tangent vector $\gamma'_v(0) = v$. The correponding exponential map is defined by $\exp_p(v) = \gamma_v(1)$. In general, the exponetial map is only locally defined, that is, it is a local differeomorphism\cite{Pennec2006} that takes a small neighborhood of the origin at $T_pM$, to a neighborhood of $p$ in the manifold. This is because it relies on the theorem of existence and uniqueness for ordinary differential equations which is local in nature. An affine connection is called complete if the exponential map is well defined at every point of the tangent bundle. \\
In short, an exponential map maps an element in the tangent space onto the manifold, the logarithm map maps an element in the manifold onto the tangent space. 


\begin{remark}
	
\end{remark}


\subsubsection{Geodesics on the Riemannian manifold}
\begin{definition}
A geodesic on a manifold is curve with zero acceleration, where acceleration is measured by calculating the usual acceleration in the ambient Euclidean space and then project orthogonally onto the tangent space.	
In terms of the local coordinate system $(x_1,x_2,...x_n)$, a curve of geodesic is required to satisfy $$\ddot{\gamma}^i + \Gamma^i_{jk}\dot{\gamma}^j\dot{\gamma}^k = 0, $$ 
where $$\Gamma^i_{jk} = \frac{1}{2} g^{im}(\partial_k g_{mj} + \partial_j g_{mk} - \partial_m g_{jk}).$$
\end{definition}
 



\subsection{Local chart due to geodesic }
\cite{Pennec2006}From above, we know that the exponential map at $p \in M$ sets up a local diffeomorphism bewteen $M$ and $T_pM$. This sets up a chart that use 

\subsection{Cut locus }
\cite{Pennec2006}It is natural to search for the maximal domain where the exponential map is diffeomorphism. If we follow a geodesic $\gamma(t) = \exp_x(t v)$ from $t = 0$ to $\infty$, it is either always minimizing all along or it is minimizing upto a time $t_0 < \infty$. In the latter case, the point $z = \gamma(t_0)$ is called a cut point and the corresponding tangent vector $t_0v$ a tangential cut point. The set of all cut points of all geodesics starting from $x$ is the cut locus $C(x) \in M$ and the set of corresponding vectors the tangential cut locus $\mathcal{C}(x)\in T_xM$. 

\subsection{Simulating the Brownian motion on Riemannian manifold }

We can simulate the Brownian motion $B(t)$ on a Riemannian manifold $M$ using following procedures:\cite{Manton2013}
\begin{itemize}
	\item Choose a starting point $p$ on $M$; set $B(0)$ to this point. Fix a step size $\delta t$.
	\item For $k=0,1,...,$ recursively do the following. Generate a Gaussian random vector $W(k) \in T_{B(k\delta t)M} \subset \mathbb{R}^n$(one way to do this is to generate an $n$-dimensional $N(0,I)$ Gaussian random vector and projecting the vector orthogonally onto $T_{B(k\delta t)}M$). Then define
	$$B(k\delta t + t) = \exp_{B(k\delta t)}(\frac{t}{\delta t}\sqrt(\delta t)W(k))$$ for $t\in [0,\delta t]$.
	\item Note that If $M = \mathbb{R}^n$, then $\exp_p(v) = p + v$.
\end{itemize}



\section{Notes on bibliography}



For differential geometry, see \cite{kreyszig2013differential}.

For Riemann geometry, see \cite{mcinerney2013first}.

For Lie group, see \cite{chirikjian2011stochastic}\cite{chirikjian2011stochastic1}\cite{frankel2004geometry}.

\printbibliography
\end{refsection}
